\begin{defi}[Fläche]
Eine \emph{Fläche} ist eine glatte 2-dimensionale Mf.
\end{defi}


\begin{defi}[Orientierbarkeit]
Eine Fläche heißt \emph{nicht orientierbar}, wenn sie eine Teilmenge enthält, die zum Möbiusband homöomorph ist.
\end{defi}

\begin{bsp}[für nicht orientierbar]
Möbiusband; $\mathbb{RP}^2$ = \{ Geraden durch $0 \in \RR^3$ \}.
\end{bsp}

\begin{bem}
	Um neue Flächen zu gewinnen kann man Polygone mit gerader Kantenzahl nehmen und 
	paarweise kanten identifizieren.
\end{bem}

\begin{bem}[Systematisches Vorgehen]
	Gehe gegen den Uhrzeigersinn die Seiten des Polygons ab und gebe jeder Seite
	ein Symbol, e.g. $a$, falls a in Durchlaufrichtung zeigt, sonst $a^-1$.
\end{bem}

\begin{bsp}
	$S^2 = aa^{-1}$, 
	Donut $T^2 = aba^{-1}b^{-1}$, 
	$\RR\mathbb{P}^2 = abab = aa$,
	Kleinsche Flasche $K^2 = aba^{-1}b^{-1}$
\end{bsp}

\begin{defi}[Zusammenhängende Summe]
	Schneide aus $F$ und $F'$ eine abgeschlossene Kreisscheibe aus, 
	identifiziere die Ränder der Scheiben mit einem Homöomorphismus. 
	$\Rightarrow$ Zusammenhängende Summe $F\#F'$.
\end{defi}

\begin{bem}
	$\#$ ist kommutativ, assoziativ und das neutrale Element ist $S^2$. 
	$(\{F\}, \#)$ ist also eine Halbgruppe.
\end{bem}

\begin{bem}
	Entspricht $F$ dem Zeichensatz $A$ und $F'$ dem Zeichensatz $B$,
	so $F\#F'$ dem Zeichensatz $AB$.
\end{bem}

\begin{bsp}
	$a_1 b_1 a_1^{-1} b_1^{-1}\dots a_g b_g a_g^{-1} b_g^{-1}$
	entspricht einer geschlossenen orientierbaren Fläche mit $g$ Löchern.
\end{bsp}

\begin{satz}[Klassifikationssatz für Flächen]
Eine geschlossene Fläche ist entweder homöomorph zu
\begin{itemize}
	\item $S^2$ oder
	\item einer zusammenhängenden Summe von $T^2$ (Tori) oder
	\item einer zusammenhängenden Summe von $\mathbb{RP}^2$.
\end{itemize}
\end{satz}

\begin{lemma}
$\mathbb{RP}^2 \# \mathbb{RP}^2 \approx K^2$ (Kleinsche Flasche).
\end{lemma}

\begin{lemma}
$3 \mathbb{RP}^2 \approx T^2 \# \mathbb{RP}^2 $ (Kleinsche Flasche).
\end{lemma}