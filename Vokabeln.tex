\documentclass[a4paper,11pt]{article} % Die Angaben in eckigen Klammern sind optional (Standardschriftgröße ist 10pt)
%%%%%%%%%%%%%%%%%%%%%%%%%%%%%%%%%%%%%%%%%%%%%%%%%%%%%%%%%%%%%%%%%%%%%%%%%%%%%%%%%%%%%%%%%%%%%%%%
%
% Vorlage für ein Aufgabenblatt im Seminar ``Digitale Werkzeuge im Mathematikunterricht'' am KIT
%
%%%%%%%%%%%%%%%%%%%%%%%%%%%%%%%%%%%%%%%%%%%%%%%%%%%%%%%%%%%%%%%%%%%%%%%%%%%%%%%%%%%%%%%%%%%%%%%%
\usepackage[english,ngerman]{babel} % So ist deutsch die Standardsprache, englisch kann eingeschaltet werden.
\usepackage[utf8]{inputenc} % Für Umlaute
\usepackage[T1]{fontenc} % Für bessere Silbentrennung
\usepackage{lmodern} % Für eine weniger pixelige Schriftart
\usepackage{graphicx} % Für Bilder (png, pdf, jpg,...)

%-------------Mathezeugs--------------
\usepackage{amssymb}
\usepackage{amsthm} % Für mehr Funktionen bei den theorem-Umgebungen
\usepackage{mathtools} % Eine Weiterentwicklung von amsmath
%
%-------Abstände etc.--------------
\usepackage{geometry} % Zur individuellen Anpassung der Seitenränder, z.B.:
\geometry{top=15mm, bottom=15mm, left=20mm, right=15mm}
%
\setlength{\parindent}{0ex} % Kein Einrücken bei einem Absatz
\renewcommand{\arraystretch}{1.5}   % Zeilenbreite bei Tabellen
\pagestyle{empty}                   % keine Seitennummern
%
%----------Kurzbefehle für Mathe-Buchstaben -------------------
\newcommand{\NN}{\mathbb N}
\newcommand{\ZZ}{\mathbb Z}
\newcommand{\QQ}{\mathbb Q}
\newcommand{\RR}{\mathbb R}
%
%------------Theorem-Definitionen----------
% braucht man für Seminararbeiten o.Ä.
\theoremstyle{definition} % Die folgenden Umgebungen werden damit nicht kursiv
\newtheorem{satz}{Satz}[section] % Die Nummerierung für Sätze richtet sich so nach Kapiteln
% Das [satz] in den folgenden Definitionen bedeutet, dass alle diese Theoreme fortlaufend mit den Sätzen nummeriert werden (Also "Satz 1.1, Bemerkung 1.2, Satz 1.3" statt "Satz 1.1, Bemerkung 1.1, Satz 1.2" etc.)
\newtheorem{prop}[satz]{Proposition}
\newtheorem{lem}[satz]{Lemma}
\newtheorem{kor}[satz]{Korollar}
\newtheorem{bem}[satz]{Bemerkung}
\newtheorem{defi}[satz]{Definition}
\newtheorem{bsp}[satz]{Beispiel}

%%%%%%%%%%%%%%%%%%%%%%%%%%%%%%%%%%%%%%%%%%%%%%%%%%%%%%%%%%%%%%%%%%%%%%%%%%%%%%%%%%%%%%%%%%%%%%%%
\usepackage{enumitem}
\setlist{nolistsep}
%
\begin{document}

\raisebox{-5mm}{\includegraphics[width=30mm]{kit_logo_de_color.jpg}}
%
\hfill \parbox{24mm}
{
Malte Voss\\
David Knothe\\
}

\rule{\textwidth}{1pt}                                   % Linie über die ganze Textbreite
%
\begin{center}
\textbf{
Vokabeln der Elementaren Geometrie \\[1ex] % Zeilenumbruch mit Abstand
}
% Vortragsnummer und Thema
{
% \Large 1. Blatt:   \glqq Thema\grqq } \\[1ex]
% Die Befehle \glqq und \grqq stehen dabei für german left / right double quote.
%
% Name der Vortragenden und Datum der Erstellung
%
\today
}
\end{center}
%
\rule{\textwidth}{1pt}\\                                 % Linie über die ganze Textbreite
%%%%%%%%%%%%%%%%%%%%%%%%%%%%%%%%%%%%%%%%%%%%%%%%%%%%%%%%%%%%%%%%%%%%%%%%%%%%%%%%%%%%%%%%%%%%%%%%
\section{Euklidische Geometrie}
\input{Vok_EG}
\vspace{3ex}
%%%%%%%%%%%%%%%%%%%%%%%%%%%%%%%%%%%%%%%%%%%%%%%%%%%%%%%%%%%%%%%%%%%%%%%%%%%%%%%%%%%%%%%%%%%%%%%%
\section{Topologie}
$\dots$

\begin{defi}[Wegzusammenhang]
Ein topologischer Raum $X$ heißt \emph{wegzusammenhängend}, wenn es zu je zwei Punkten $a, b \in X$ einen Weg von $a$ nach $b$ gibt, d.h. eine stetige Abbildung $f: [0, 1] \to X$ mit $f(0) = a$, $f(1) = b$.
\end{defi}

\begin{bem} Es gilt:
\begin{itemize}
\item Wegzusammenhängende Räume sind zusammenhängend.
\item Nichtdisjunkte Vereinigungen von (weg-)zusammenhängenden Räumen sind (weg-)zusammenhängend
\item $X \times Y$ wegzusammenhängend $\Leftrightarrow$ $X$ und $Y$ jeweils wegzusammenhängend
\item Stetige Bilder von (weg-)zusammenhängenden Mengen sind (weg-)zusammenhängend. \\
\end{itemize}
\end{bem}


\begin{defi}[Hausdorffsch]
Ein topologischer Raum $X$ heißt \emph{hausdorffsch}, wenn es zu je zwei verschiedenen Punkten $x, y \in X$ disjunkte Umgebungen von $x$ und $y$ gibt.
\end{defi}

\begin{bem} Jeder metrische Raum ist hausdorffsch. \\
\end{bem}

\begin{defi}[Kompaktheit]
Eine Menge / topologischer Raum $X$ heißt \emph{kompakt}, wenn jede offene Überdeckung von $X$ eine endliche Teilüberdeckung besitzt.
\end{defi}

\begin{bem} $X \neq \emptyset \neq Y$ sind kompakt $\Leftrightarrow$ $X + Y$ kompakt $\Leftrightarrow$ $X \times Y$ kompakt.
\end{bem}

\begin{bsp} Es gilt:
\begin{itemize}
\item Auf der diskreten Topologie gilt: $X$ kompakt $\Leftrightarrow$ $X$ endlich.
\item Abgeschlossene Intervalle in $\RR$ sind kompakt.
\item Alle abgeschlossenen und beschränkten Teilmengen des $\RR^N$ sind kompakt.
\item Abgeschlossene Teilmengen von einem Kompaktum sind kompakt. \\
\end{itemize}
\end{bsp}

\begin{satz}[von Heine-Borel] Die kompakten Teilmengen des $\RR^n$ sind genau die abgeschlossenen und beschränkten.
\end{satz}

\begin{kor}
Stetige Funktionen auf Kompakta sind beschränkt.
\end{kor}
\begin{kor}
$X$ Hausdorff-Raum, $K \subseteq X$ kompakt, so $K$ abgeschlossen in $X$. \\
\end{kor}

\paragraph*{Schlüsse von lokalen auf globale Eigenschaften}

$\dots$

\vspace{3ex}
%%%%%%%%%%%%%%%%%%%%%%%%%%%%%%%%%%%%%%%%%%%%%%%%%%%%%%%%%%%%%%%%%%%%%%%%%%%%%%%%%%%%%%%%%%%%%%%%
\section{Mannigfaltigkeiten}
\input{Vok_Mf}


%%%%%%%%%%%%%%%%%%%%%%%%%%%%%%%%%%%%%%%%%%%%%%%%%%%%%%%%%%%%%%%%%%%%%%%%%%%%%%%%%%%%%%%%%%%%%%%%

\vspace*{\fill} % vertikaler Abstand, der auffüllt, \vfill ginge auch

\begin{flushright}
\textit{ Besprechung der Aufgaben ab dem 21.10.2019 }
\end{flushright}


\end{document} % Alles dahinter wird wie ein Kommentar behandelt!
%
% Formeln im Text beginen und enden mit dem Dollar-Zeichen: $
%
% eine abgesetzte Formel beginnt mit \[ und endet mit \] oder
%
\begin{equation}

\end{equation}
%
% Auzählungen mit Nummern:
\begin{enumerate}
 \item[a)]
 \item[b)]
 \item[c)]
\end{enumerate}
%
%Aufzählung ohne Nummern
\begin{itemize}
 \item
 \item
 \item
\end{itemize}
%
% Einbinden von Bildern:
\includegraphics[width=  ]{  }
%
% Tabelle mit drei Spalten (link, rechts, zentriert und Trennlinien
% Spalten werden durch & getrennt
\begin{tabular}{|l|c|r|} \hline
 & & \\ \hline
\end{tabular}
