\documentclass[a4paper,11pt]{article} % Die Angaben in eckigen Klammern sind optional (Standardschriftgröße ist 10pt)
%%%%%%%%%%%%%%%%%%%%%%%%%%%%%%%%%%%%%%%%%%%%%%%%%%%%%%%%%%%%%%%%%%%%%%%%%%%%%%%%%%%%%%%%%%%%%%%%
%
% Vorlage für ein Aufgabenblatt im Seminar ``Digitale Werkzeuge im Mathematikunterricht'' am KIT
%
%%%%%%%%%%%%%%%%%%%%%%%%%%%%%%%%%%%%%%%%%%%%%%%%%%%%%%%%%%%%%%%%%%%%%%%%%%%%%%%%%%%%%%%%%%%%%%%%
\usepackage[english,ngerman]{babel} % So ist deutsch die Standardsprache, englisch kann eingeschaltet werden.
\usepackage[utf8]{inputenc} % Für Umlaute
\usepackage[T1]{fontenc} % Für bessere Silbentrennung
\usepackage{lmodern} % Für eine weniger pixelige Schriftart
\usepackage{graphicx} % Für Bilder (png, pdf, jpg,...)

%-------------Mathezeugs--------------
\usepackage{amssymb}
\usepackage{amsthm} % Für mehr Funktionen bei den theorem-Umgebungen
\usepackage{mathtools} % Eine Weiterentwicklung von amsmath
\usepackage{hyperref}
%
%-------Abstände etc.--------------
\usepackage{geometry} % Zur individuellen Anpassung der Seitenränder, z.B.:
\geometry{top=15mm, bottom=15mm, left=20mm, right=15mm}
%
\setlength{\parindent}{0ex} % Kein Einrücken bei einem Absatz
\renewcommand{\arraystretch}{1.5}   % Zeilenbreite bei Tabellen
\pagestyle{empty}                   % keine Seitennummern
%
%----------Kurzbefehle für Mathe-Buchstaben -------------------
\newcommand{\NN}{\mathbb N}
\newcommand{\ZZ}{\mathbb Z}
\newcommand{\QQ}{\mathbb Q}
\newcommand{\RR}{\mathbb R}
%
%------------Theorem-Definitionen----------
% braucht man für Seminararbeiten o.Ä.
\newtheoremstyle{costum}
  {\topsep}   % ABOVESPACE
  {\topsep}   % BELOWSPACE
  {\normalfont}  % BODYFONT
  {0pt}       % INDENT (empty value is the same as 0pt)
  {\itshape} % HEADFONT
  {.}         % HEADPUNCT
  {5pt plus 1pt minus 1pt} % HEADSPACE
  {\thmname{#1}\thmnumber{ #2}\thmnote{\textnormal{ \textbf{#3}}}}          % CUSTOM-HEAD-SPEC
\theoremstyle{costum} % Die folgenden Umgebungen werden damit nicht kursiv
\newtheorem{satz}{Satz}[section] % Die Nummerierung für Sätze richtet sich so nach Kapiteln
% Das [satz] in den folgenden Definitionen bedeutet, dass alle diese Theoreme fortlaufend mit den Sätzen nummeriert werden (Also "Satz 1.1, Bemerkung 1.2, Satz 1.3" statt "Satz 1.1, Bemerkung 1.1, Satz 1.2" etc.)
% \newtheorem{prop}[satz]{Proposition}
% \newtheorem{lem}[satz]{Lemma}
\newtheorem*{kor}{Korollar}
\newtheorem*{bem}{Bem}
\newtheorem*{defi}{Def}
\newtheorem*{bsp}{Bsp}
\newtheorem*{lemma}{Lemma}

%%%%%%%%%%%%%%%%%%%%%%%%%%%%%%%%%%%%%%%%%%%%%%%%%%%%%%%%%%%%%%%%%%%%%%%%%%%%%%%%%%%%%%%%%%%%%%%%
\usepackage{enumitem}
\setlist{nolistsep}
%
\begin{document}

\raisebox{-5mm}{\includegraphics[width=30mm]{Bilder/kit_logo_de_color.jpg}}
%
\hfill \parbox{24mm}
{
Malte Voss\\
David Knothe\\
}

\rule{\textwidth}{1pt}                                   % Linie über die ganze Textbreite
%
\begin{center}
\textbf{
Vokabeln der Elementaren Geometrie \\[1ex] % Zeilenumbruch mit Abstand
}
% Vortragsnummer und Thema
{
% \Large 1. Blatt:   \glqq Thema\grqq } \\[1ex]
% Die Befehle \glqq und \grqq stehen dabei für german left / right double quote.
%
% Name der Vortragenden und Datum der Erstellung
%
\today, \url{https://github.com/vossmalte/ElementareGeometrie}
}
\end{center}
%
\rule{\textwidth}{1pt}\\                                 % Linie über die ganze Textbreite
%%%%%%%%%%%%%%%%%%%%%%%%%%%%%%%%%%%%%%%%%%%%%%%%%%%%%%%%%%%%%%%%%%%%%%%%%%%%%%%%%%%%%%%%%%%%%%%%
\section{Euklidische Geometrie}
\input{Subfiles/Einfuehrung_Geometrie}
\vspace{3ex}
%%%%%%%%%%%%%%%%%%%%%%%%%%%%%%%%%%%%%%%%%%%%%%%%%%%%%%%%%%%%%%%%%%%%%%%%%%%%%%%%%%%%%%%%%%%%%%%%
\section{Topologie}
\begin{defi}[Topologie]
  Sei $(X,\mathcal{O})$ mit $\mathcal{O} \subset 2^X$.
  $U \in \mathcal{O}$ nennt man offen.
  $\mathcal{O}$ heißt Topologie auf $X$, falls gelten:
  \begin{enumerate}
    \item $X, \emptyset \in \mathcal{O}$
    \item Beliebige Vereinigungen offener Mengen sind offen
    \item Endliche Schnitte offener Mengen sind offen
  \end{enumerate}
\end{defi}

\begin{defi}[Stetigkeit]
  $f:X\rightarrow Y$ heißt stetig, wenn die Urbilder offener Mengen stets offen sind.
\end{defi}

\begin{defi}[Abgeschlossen]
  $A \subset X$ abgeschlossen $:\Leftrightarrow X\backslash A$ offen
\end{defi}
\begin{defi}[Umgebung]
  $U \subset X$ heißt Umgebung von $x\in X$, falls
  $\exists V\in \mathcal{O}$ mit $x\in V \subset U$
\end{defi}
\begin{defi}[Innen, außen, Rand, Abschluss]
  ...
\end{defi}

\begin{defi}[Topologie eines metrischen Raums]
  Jeder metrische Raum ist ein topologischer Raum.
  Die induzierte Topologie $\mathcal{O}(d)$ geht so: $V$ heißt offen,
  wenn für jedes $x\in V$ ein $\varepsilon$-Ball um $x$ in $V$ liegt.
\end{defi}
\begin{defi}[$\mathcal{O}$ metrisierbar]
  $:\Leftrightarrow \exists d: \mathcal{O} = \mathcal{O}(d)$
\end{defi}

\begin{defi}[fein, grob]
  $\mathcal{O} \subset \mathcal{O}' \Leftrightarrow$
  $\mathcal{O}'$ feiner als $\mathcal{O}$
\end{defi}

\begin{defi}[disjunkte Summe / Vereinigung]
  von $X, Y$ erklärt als
  $X+Y = X\times Y:= X\times {0} \cup Y\times {1}$
\end{defi}

\begin{defi}[Topologische Summe]
  $(X,\mathcal{O}), (Y,\mathcal(O)')$ top. Räume.
  $\{U+V | U \in \mathcal{O}, V \in \mathcal{O}'\}$
  ist Topologie auf $X+Y$. $X+Y$ dann top. Summe.
\end{defi}

\begin{defi}[Produkttopologie]
  $W \subset X\times Y$ offen
  $:\Leftrightarrow \forall (x,y) \in W: \exists$
  Umgebung $U$ von $x$ in X und Umgebung $V$ von $y$ in Y
  mit $U \times V \subset W$.
\end{defi}

\begin{bem}
  Solche \glqq Rechtecke\grqq{} sind offen in der Produkttopologie,
  aber nicht alle offenen Mengen sind solche
\end{bem}


$\dots$

\begin{defi}[Wegzusammenhang]
Ein topologischer Raum $X$ heißt \emph{wegzusammenhängend}, wenn es zu je zwei Punkten $a, b \in X$ einen Weg von $a$ nach $b$ gibt, d.h. eine stetige Abbildung $f: [0, 1] \to X$ mit $f(0) = a$, $f(1) = b$.
\end{defi}

\begin{bem} Es gilt:
\begin{itemize}
\item Wegzusammenhängende Räume sind zusammenhängend.
\item Nichtdisjunkte Vereinigungen von (weg-)zusammenhängenden Räumen sind (weg-)zusammenhängend
\item $X \times Y$ wegzusammenhängend $\Leftrightarrow$ $X$ und $Y$ jeweils wegzusammenhängend
\item Stetige Bilder von (weg-)zusammenhängenden Mengen sind (weg-)zusammenhängend. \\
\end{itemize}
\end{bem}


\begin{defi}[Hausdorffsch]
Ein topologischer Raum $X$ heißt \emph{hausdorffsch}, wenn es zu je zwei verschiedenen Punkten $x, y \in X$ disjunkte Umgebungen von $x$ und $y$ gibt.
\end{defi}

\begin{bem} Jeder metrische Raum ist hausdorffsch. \\
\end{bem}

\begin{defi}[Kompaktheit]
Eine Menge / topologischer Raum $X$ heißt \emph{kompakt}, wenn jede offene Überdeckung von $X$ eine endliche Teilüberdeckung besitzt.
\end{defi}

\begin{bem} $X \neq \emptyset \neq Y$ sind kompakt $\Leftrightarrow$ $X + Y$ kompakt $\Leftrightarrow$ $X \times Y$ kompakt.
\end{bem}

\begin{bsp} Es gilt:
\begin{itemize}
\item Auf der diskreten Topologie gilt: $X$ kompakt $\Leftrightarrow$ $X$ endlich.
\item Abgeschlossene Intervalle in $\RR$ sind kompakt.
\item Alle abgeschlossenen und beschränkten Teilmengen des $\RR^N$ sind kompakt.
\item Abgeschlossene Teilmengen von einem Kompaktum sind kompakt.
\end{itemize}
\end{bsp}

\begin{bem}
Kompaktheit erlaubt \emph{Schlüsse von lokalen auf globale Eigenschaften}. Bspw.:
\begin{itemize}
	\item Ist $f: X \to \RR$ (stetig) lokal beschränkt, so auch global.
	\item Ist $\{A_i\}$ eine lokal endliche Überdeckung von X, so ist die Überdeckung endlich. \\
\end{itemize}
\end{bem}

\begin{satz}[von Heine-Borel] Die kompakten Teilmengen des $\RR^n$ sind genau die abgeschlossenen und beschränkten.
\end{satz}

\begin{kor}
Stetige Funktionen auf Kompakta sind beschränkt.
\end{kor}
\begin{kor}
$X$ Hausdorff-Raum, $K \subseteq X$ kompakt, so $K$ abgeschlossen in $X$. \\
\end{kor}


\begin{defi}[Limes]
$(x_n)$ Folge in $X$. Dann ist $a \in X$ ist \emph{Limes} der Folge, wenn sich in jeder Umgebung U von $a$ fast alle Folgenglieder befinden.
\end{defi}
\begin{bem}
In Hausdorff-Räumen sind Limiten eindeutig.
\end{bem}

\vspace{3ex}
%%%%%%%%%%%%%%%%%%%%%%%%%%%%%%%%%%%%%%%%%%%%%%%%%%%%%%%%%%%%%%%%%%%%%%%%%%%%%%%%%%%%%%%%%%%%%%%%
\section{Mannigfaltigkeiten}
\begin{defi}[Topologische Mannigfaltigkeit]
Eine \emph{topologische Manngifaltigkeit} $M$ ist ein Hausdorff-Raum mit abzählbarer Basis der Topologie, sodass jeder Punkt $p$ aus $M$ eine offene Umgebung besitzt, die homöomorph zu einer offenen Menge von $\RR^N$ ist. ("die lokal so aussieht wie der $\RR^N$")
\end{defi}

\begin{defi}[Karte, Atlas]
Ein solcher Homöomorphismus $\phi: U \to V \subseteq \RR^N$ heißt \emph{Karte (um $p$)}; ein \emph{Atlas} von $M$ ist eine Menge $\mathcal{A}$ von Karten von M, deren Definitionsbereiche ganz $M$ überdecken.
\end{defi}

\begin{defi}[Kartenwechsel]
Für $(U, \phi)$ und $(V, \psi)$ aus $\mathcal{A}$ heißt \underline{$\psi \circ \phi^{-1}$} $: \phi(U \cap V) \to \psi(U \cap V)$ \emph{Kartenwechsel}.
\end{defi}

\begin{defi}[glatt, maximal, $C^\infty$]
Ein $C^{\infty}$-Atlas ist ein Atlas auf $M$, für den alle Kartenwechsel $C^{\infty}$-Diffeo\-morphismen sind, und er heißt \emph{maximal}, wenn er in keinem anderen echt enthalten ist.
In diesem Fall nennt man $\mathcal{A}$ auch \emph{differenzierbare} oder \emph{glatte Struktur} auf $M$ und nennt $M$ \emph{differenzierbare Mannigfaltigkeit}.
\end{defi}

\begin{bsp} Beispiele für glatte Mannigfaltigkeiten:
\begin{itemize}
	\item $\RR^N$ und alle offenen Teilmengen
	\item $S^1$
	\item Produkte $M \times N$ von glatten Mannigfaltigkeiten
	\item $T^N := S^1 \times \dots \times S^1$, der \emph{N-dimensionale Torus}
	\item GL$(n, \RR) = \det^{-1}(\RR \backslash \{0\})$
\end{itemize}	
\end{bsp}

\begin{bsp}[Stereographische Projektion]
	Die Quotientenabbildung $\RR^{n+1}\backslash\{0\} \to \RR \mathbb{P}^n$
	bildet die $S^n\subset \RR^{n+1}$ surjektiv auf $\RR \mathbb{P}^n$ ab.\\
	$\Leftarrow\RR \mathbb{P}^n$ ist kompakt als stetiges Bild eines Kompaktums.
\end{bsp}

\vspace{3ex}
%%%%%%%%%%%%%%%%%%%%%%%%%%%%%%%%%%%%%%%%%%%%%%%%%%%%%%%%%%%%%%%%%%%%%%%%%%%%%%%%%%%%%%%%%%%%%%%%
\section{Projektive Räume}
\begin{defi}[Projektiver Raum]
Wir definieren $\mathbb{RP}^N = P(\RR^{N+1})$, den \emph{n-dimensionalen reellen projektiven Raum}, als die Menge der Äquivalenzklassen von Elementen aus $\RR^{N+1}$ unter der Äquivalenzrelation $\sim$: $x \sim y \Leftrightarrow \exists \lambda \in \RR \backslash \{0\}: x = \lambda \cdot y.$
\end{defi}

\begin{satz}
Der $\mathbb{RP}^N$ ist eine N-dimensionale differenzierbare Mannigfaltigkeit.
\end{satz}

\begin{bem}
Der $\mathbb{CP}^N = P(\mathbb{C}^{N+1})$ wird mit der gleichen Äquivalenzrelation wie oben definiert, nur dass jetzt $\lambda \in \mathbb{C} \backslash \{0\}$.
Der $\mathbb{CP}^N$ ist eine \emph{reelle} (!) $2N$-dimensionale differenzierbare Mannigfaltigkeit.
\end{bem}

\begin{bem}[Isomorphie zur $S^N$] Es gilt:
\begin{itemize}
	\item $\mathbb{RP}^N \cong S^N / \sim$ mit $x \sim y \Leftrightarrow x = \pm y$.
	\item $\mathbb{CP}^N \cong S^{2N+1} / \sim$ mit $x \sim y \Leftrightarrow x = \lambda \cdot y$ für $\lambda \in S^1$.
\end{itemize}
\end{bem}
\vspace{3ex}
%%%%%%%%%%%%%%%%%%%%%%%%%%%%%%%%%%%%%%%%%%%%%%%%%%%%%%%%%%%%%%%%%%%%%%%%%%%%%%%%%%%%%%%%%%%%%%%%
\section{Vektorfelder}
\dots

\begin{defi}[Vollständiges VF]
    Ein Vektorfeld $X$ auf $M$ hei§t \emph{vollständig}, 
    wenn es einen globalen Fluss $\Phi$ besitzt.
\end{defi}

\begin{defi}[Riemannsche Mf]
    Eine \emph{riemannsche Metrik} $g$ auf einer glatten Mf $M$ ordnet 
    jedem Punkt $p \in M$ ein Skalarprodukt $g(p) = g_p$ auf $T_pM$ zu, sodass:
    für je zwei glatte VFer $X, Y \in \mathcal{X}(M)$ ist $g(X, Y)$ glatt, 
    wobei $g(X, Y) := g_p(X_p, Y_p)$. \\
    Eine \emph{riemannsche Mannigfaltigkeit} ist eine glatte Mf mit einer riemannschen Metrik.
\end{defi}

\vspace{3ex}
%%%%%%%%%%%%%%%%%%%%%%%%%%%%%%%%%%%%%%%%%%%%%%%%%%%%%%%%%%%%%%%%%%%%%%%%%%%%%%%%%%%%%%%%%%%%%%%%
\section{Flächen}
\begin{defi}[Fläche]
Eine \emph{Fläche} ist eine glatte 2-dimensionale Mf.
\end{defi}

\begin{bsp}
$S^2$, Donut, \dots
\end{bsp}

\begin{defi}[Orientierbarkeit]
Eine Fläche heißt \emph{nicht orientierbar}, wenn sie eine Teilmenge enthält, die zum Möbiusband homöomorph ist.
\end{defi}

\begin{bsp}
Möbiusband; $\mathbb{RP}^2$ = \{ Geraden durch $0 \in \RR^3$ \}.
\end{bsp}

\begin{satz}[Klassifikationssatz für Flächen]
Eine geschlossene Fläche ist entweder homöomorph zu
\begin{itemize}
	\item $S^2$ oder
	\item einer zusammenhängenden Summe von $T^2$ (Tori) oder
	\item einer zusammenhängenden Summe von $\mathbb{RP}^2$.
\end{itemize}
\end{satz}

\begin{lemma}
$\mathbb{RP}^2 \# \mathbb{RP}^2 \approx K^2$ (Kleinsche Flasche).
\end{lemma}

\begin{lemma}
$3 \mathbb{RP}^2 \approx T^2 \# \mathbb{RP}^2 $ (Kleinsche Flasche).
\end{lemma}
\vspace{3ex}
%%%%%%%%%%%%%%%%%%%%%%%%%%%%%%%%%%%%%%%%%%%%%%%%%%%%%%%%%%%%%%%%%%%%%%%%%%%%%%%%%%%%%%%%%%%%%%%%

\vspace*{\fill} % vertikaler Abstand, der auffüllt, \vfill ginge auch

\begin{flushright}
\end{flushright}


\end{document} % Alles dahinter wird wie ein Kommentar behandelt!
